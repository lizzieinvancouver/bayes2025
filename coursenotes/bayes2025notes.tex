\documentclass[11pt]{article}
\usepackage[top=1.00in, bottom=1.0in, left=1in, right=1in]{geometry}
\renewcommand{\baselinestretch}{1.1}
\usepackage{graphicx}
\usepackage{natbib}
\usepackage{amsmath}
\usepackage{hyperref}
\usepackage{parskip}
\usepackage{todonotes}


\def\labelitemi{--}
\parindent=0pt

\begin{document}
\bibliographystyle{/Users/Lizzie/Documents/EndnoteRelated/Bibtex/styles/besjournals}
\renewcommand{\refname}{\CHead{}}

% To do moved to git issue: https://github.com/temporalecologylab/bayes2025/issues/1
\section{Class 1} 

\subsection{Structure of today}
\begin{itemize}
\item Few minutes for urgent course content questions ... 
\item Review what is Bayesian (briefly) and the workflow I will teach for it
\item Focus on simulating data from a linear regression
\item END: course content, grading etc..
\end{itemize}
Okay! You all should have received my email about the course, which means\\ {\bf you're here because ... }
\begin{itemize}
\item Excited to learn Bayesian inference and modeling!
\item Excited to work together in pairs or teams during class (even if you're auditing)
\item Know enough R to code actively in class 
\item Have a laptop and note-taking devices
\end{itemize}
If you're {\bf not sure} about any of these, stay here and come talk with me after. 

\subsection{Who I am and course aims ...}

{\bf Who am I?} Quick review of how I learned Bayesian and how it is all I basically used now. Like many died in the wool Bayesians I believe this makes me: 
\begin{itemize}
\item Happier, free from p-values
\item Think harder about the science
\item Have a WAY better sense of how the models I use work and how well they work on data similar to my own. 
\item More suspicious of a lot of stats
\end{itemize}

\subsection{Before I dive in ... reminder: don't panic} % Don't panic: How to get the most out of a stats class
No one gets everything in a stats class the first time, but you need to keep listening and not zone out.

\subsection{What I want you to get out of this class} 
\begin{itemize}
\item The basics of what Bayesian is and how to implement
\item The importance of a workflow and understand one I use and recommend
\item Give some example of what they'll be able to do at the end (will vary by student)
\item Get some of your burning questions answered ... {\bf Feel free to ask questions/interrupt!}
\item You probably will not come out of this class ready to analyze cut-point ordinal models for your community ecology data, but you'll have the workflow skills to start to think about to approach such a problem. 
\end{itemize}

\subsection{What is Bayesian? Pros and cons}
It's a way of getting estimates from a model based on the likelihood from data and your prior beliefs.\\
It's a way of fitting and inferring from models that is extremely flexible and relies on prior knowledge. (That's basically all you need to know for today.)\\

\todo[inline]{Ask the students to list out pros and cons. Make sure they hit the below.}

Pros
\begin{itemize}
\item Very flexible!
\item Optimally handles uncertainty
\item Intuitive
\item No assumptions! No iid, nothing to memorize!
\item Stop worrying about what your p-value is or contorting yourself to accurately define a CI
\item Get mechanistic insights!
\item Have a better sense of your parameter estimates
\end{itemize}

Cons
\begin{itemize}
\item No assumptions, you must check your own model and know what you're doing ... 
\item Computationally heavy
\end{itemize}

\subsubsection{Types of Bayesians}
There are {\bf many} types of Bayesians:
\begin{itemize}
\item Andy Royle Bayesians with specific beliefs about how you fit mark-recapture models
\item People obsessed with DAGs
\item Facultative Bayesians
\item Andrew Gelman Bayesians (BDA)
\end{itemize}

{\bf I will teach you my style of Bayesian ...} which is pretty close to a Gelman Bayesian with other ideas (Betancourt etc.) thrown in. 

{\bf This does not matter! Except when you go out into the world} and meet the other Bayesians.

\subsection{What is Bayesian? A workflow}

\begin{enumerate}
\item Come up with your model
\item Simulate data from your model to check it
\item Prior predictive checks
\item Run your model on empirical data
\item Retrodictive checks (aka PPCs)
\end{enumerate}

\emph{This class will focus on most of this workflow!} \\Except step 1 and we won't dwell on step 2 (prior checks).

\subsection{Simulate from a linear model: Part 1}

{\bf We're going to use something that works with linear regression for our model, so continuous $x$ and continuous $y$}
\todo[inline]{Get class to come up with an example and DRAW it out on a graph} Options: Plant growth in response to soil nutrient concentration, biometric scaling etc. 

\todo[inline]{Ask students equation for a line.}
Write out various notations and differentiate {\bf {\Large parameters}} from {\bf \Large data}} (ideally, skip the error here)

\emph{Okay, I want to simulate data from this equation, what do I do?}

In this section be sure to ...
\begin{itemize}
\item Slope versus intercept
\item Come up with parameter numbers to write on board
\item Mention \verb|rnorm|
\item Get the ERROR onto the equation if you have not already
\item mention $n$
\item What is an effect size? 
\end{itemize}

\todo[inline]{Students should pair up and work on doing this with the following rules ...} 
\begin{itemize}
\item You must BOTH end up with the code you come up with.
\item Simulate, plot and then try to figure out a way to tell if you have done it right... 
\item You alert me when you're done, stuck or have a question ... [If they are done, they should check their work using \verb|lm|, then try to simulate a LOGISTIC regression.]
\end{itemize}

\todo[inline]{Note to self: Give the class a 10 minute break by 3pm!} 

\subsection{Simulate from a linear model: Part 2}

Come together and review how they did. Live code with them the course example using \verb|lm| and \verb|stan_glm| to check work. 

\todo[inline]{Discuss: How might this be valuable?} 
And be sure to discuss why this is critical in Bayesian approaches ...\\
NO assumptions; you must CHECK and UNDERSTAND your model.

\subsection{Simulate from a linear model: Add interactions}

Review this if time allows .... 
\begin{itemize}
\item Intercept only model
\item Adding an interaction to a model ... 
\end{itemize}

Go through the math on the board, introduce dummy variables and then set them to try to simulate a model with an interaction and see if they can return the parameters. \\
Or, we get to this tomorrow more likely ... 

\subsection{Review of course (by 4:20pm)}
\begin{itemize}
\item 3 weeks, 6 classes, we'll get to hierarchical modeling
\item ... but I am not sure when! I reserve the right to move things around (small chance I will start hierarchical modeling next week). 
\item Grading is participation and homework
\item There are TWO homework (end of each of the first 2 weeks). Please do them! Even if you are auditing.
\item No project, you must use a provided dataset
\item Course managed on GitHub; you can submit homework on GitHub or Canvas. 
\item GitHub has wiki with resources ... Review (if time allows)
\item We will use rstanarm, which is a version of Stan -- make sure you have it running before the next class. 
\item Remind me to give you a BREAK in the middle of class
\item Questions?
\end{itemize}

%%%%%%%%%%%%
%%%%%%%%%%%%
\newpage
\section{Class 2} 

Stuff to have prepped for this class ....
\begin{itemize}
\item TWO articles to show ...
\item \url{https://www.countbayesie.com/blog/2015/2/18/hans-solo-and-bayesian-priors}
\item The html from \url{https://github.com/lizzieinvancouver/bayesianflowsexample}
\item \url{https://chi-feng.github.io/mcmc-demo/app.html?algorithm=RandomWalkMH&target=banana}
\end{itemize}

{\bf Review the workflow!} Write it up  and point to where we are \\
...maybe tell them -- yes! 
You will spend more of your life fitting models to not your empirical data if you properly use the Bayesian workflow.\\
Give TWO examples of this in papers (post code for the test data later). \\ % Flynn & Wolkovich, Ettinger 2020, Morales-Castilla (nice, because I can show the pmm repo)
Maybe touch on -- I have ended up with simpler models. \\

{\bf Review equations from yesterday ...}
Be sure to encourage them to move towards the one without $normal(0, \sigma)$  use $normal(\mu, \sigma)$

\subsection{What is Bayesian: Posterior}
Go over it. Maybe on the chalkboard ...  

Use the webpage eventually to come up with an example
\href{https://www.countbayesie.com/blog/2015/2/18/hans-solo-and-bayesian-priors}{Give my Star Wars example.}

\todo[inline]{Discuss in pairs: Another example (if time allows)} 

Other examples: Complete separation.... Dolph's Iraq war example (which is not great).

\subsection{What is Bayesian: Prior}

Types of priors (informative, non-informative, weakly informative)

\todo[inline]{Discuss in pairs: An example of a prior you would set on a parameter related to your system (or just a fun example). (Only time allows)} 
Round robin of how would you set a prior for your data ... (set model first).

\subsection{How much do priors matter?}
It depends ..... ask the class what they think it depends on.

\todo[inline]{... ask the class what they think it depends on.} 

EXAMPLE: Show code where likelihood overwhelms prior. Can go through Stan briefly if time allows and introduce MCMC. 

EXAMPLE: Show prior predictive check from \url{https://github.com/lizzieinvancouver/bayesianflowsexample/blob/main/example.html} 

Why we will not do prior checks ... \\ Because they are annoying in rstanarm, brms etc. They are easier in raw Stan code. 

\subsection{MAYBE: Simulate another example ... and fit it in rstanarm} 

\subsection{What is MCMC? And why do we need it ...}

To get a posterior in Bayesian, we generally get samples from it. \\
Sampling for Bayesian models almost always have 3 ingredients:

\begin{enumerate}
\item Monte Carlo -- process to generate random draws (\verb|rnorm|)
\item Markov chain -- Monte carlo with correlated steps
\item Algorithm -- e.g., Metropolis Hastings
\end{enumerate}

ON THE BOAD: give an example for linear regression (alpha, beta, sigma) and walk through for Metropolis Hastings. \\
NEXT: Discuss proposal issue and mention GIBBS.\\
THEN: mention Stan and what it does conceptually. 

Then spend a while looking at: \url{https://chi-feng.github.io/mcmc-demo/app.html?algorithm=RandomWalkMH&target=banana}

Maybe mention `hill-climbing' -- which is what this is in some ways. 

{\bf How many dimensions is a posterior?} As many dimensions as the number of parameters you have. \\
So it's a complex space to search! \\
Mention GIBBS and how it works (long runs, search for chain hangups) versus Stan (divergences).\\ 



% \subsection{Cover interactions! If you have not already ...}
% Go through the math on the board, introduce dummy variables and then set them to try to simulate a model with an interaction and see if they can return the parameters. 


\subsection{Review the homework assignment!}
\begin{itemize}
\item Go over the tasks (homework on board)
\item Review the datasets briefly (mention hierarchical)
\item How to submit
\begin{itemize}
\item GitHub -- do you all want write access?
\item Canvas (but I prefer GitHub)
\end{itemize}
\item What to do if you get stuck ... ask classmates for help, use Piazza, move onto next step. 
\item A note on using ChatGPT
\end{itemize}

\newpage
\section{Class 3} 

{\bf Review the workflow!} Maybe write it up fast or such and point to where we are ...maybe tell them -- yes! You will spend more of your life fitting models to not your empirical data if you properly use the Bayesian workflow.

{\bf Review the homework!} 
\begin{itemize}
\item What was hard? 
\item What parameters fit better or worse? 
\item Get to what they learned about interactions ... (16X)
\end{itemize}
\end{document}


\begin{enumerate}
\item
\end{enumerate}


\begin{itemize}
\item
\end{itemize}
