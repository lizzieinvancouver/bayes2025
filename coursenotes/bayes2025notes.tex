\documentclass[11pt]{article}
\usepackage[top=1.00in, bottom=1.0in, left=1in, right=1in]{geometry}
\renewcommand{\baselinestretch}{1.1}
\usepackage{graphicx}
\usepackage{natbib}
\usepackage{amsmath}
\usepackage{hyperref}
\usepackage{parskip}
\usepackage{todonotes}


\def\labelitemi{--}
\parindent=0pt

\begin{document}
\bibliographystyle{/Users/Lizzie/Documents/EndnoteRelated/Bibtex/styles/besjournals}
\renewcommand{\refname}{\CHead{}}

% To do moved to git issue: https://github.com/temporalecologylab/bayes2025/issues/1
\section{Class 1} 

\subsection{Structure of today}
\begin{itemize}
\item Few minutes for urgent course content questions ... 
\item Review what is Bayesian (briefly) and the workflow I will teach for it
\item Focus on simulating data from a linear regression
\item END: course content, grading etc..
\end{itemize}
Okay! You all should have received my email about the course, which means\\ {\bf you're here because ... }
\begin{itemize}
\item Excited to learn Bayesian inference and modeling!
\item Excited to work together in pairs or teams during class (even if you're auditing)
\item Know enough R to code actively in class 
\item Have a laptop and note-taking devices
\end{itemize}
If you're {\bf not sure} about any of these, stay here and come talk with me after. 

\subsection{Before I dive in ... reminder: don't panic} % Don't panic: How to get the most out of a stats class
No one gets everything in a stats class the first time, but you need to keep listening and not zone out.

(Also, should I mention that this is the FIRST year I am teaching this class so bear with me?)

\subsection{What is Bayesian? Pros and cons}
It's a way of getting estimates from a model based on the likelihood from data and your prior beliefs.\\
It's a way of fitting and inferring from models that is extremely flexible and relies on prior knowledge. (That's basically all you need to know for today.)\\

\todo[inline]{Ask the students to list out pros and cons. Make sure they hit the below.}

Pros
\begin{itemize}
\item Very flexible!
\item Optimally handles uncertainty
\item Intuitive
\item No assumptions! No iid, nothing to memorize!
\end{itemize}

Cons
\begin{itemize}
\item No assumptions, you must check your own model and know what you're doing ... 
\item Computationally heavy
\end{itemize}

\subsubsection{Types of Bayesians}
There are {\bf many} types of Bayesians:
\begin{itemize}
\item Andy Royle Bayesians with specific beliefs about how you fit mark-recapture models
\item People obsessed with DAGs
\item Facultative Bayesians
\item Andrew Gelman Bayesians (BDA)
\end{itemize}

{\bf I will teach you my style of Bayesian ...} which is pretty close to a Gelman Bayesian with other ideas (Betancourt etc.) thrown in. 

{\bf This does not matter! Except when you go out into the world} and meet the other Bayesians.

\subsection{What is Bayesian? A workflow}

\begin{enumerate}
\item Come up with your model
\item Simulate data from your model to check it
\item Prior predictive checks
\item Run your model on empirical data
\item Retrodictive checks (aka PPCs)
\end{enumerate}

\emph{This class will focus on most of this workflow!} \\Except step 1 and we won't dwell on step 2 (prior checks).

\section{Simulate from a linear model: Part 1}

{\bf We're going to use something that works with linear regression for our model, so continuous $x$ and continuous $y$}
\todo[inline]{Get class to come up with an example and DRAW it out on a graph} Options: Plant growth in response to soil nutrient concentration, biometric scaling etc. 

\todo[inline]{Ask students equation for a line.}
Write out various notations and differentiate {\bf {\Large parameters}} from {\bf \Large data}} (ideally, skip the error here)

\emph{Okay, I want to simulate data from this equation, what do I do?}

In this section be sure to ...
\begin{itemize}
\item Slope versus intercept
\item Come up with parameter numbers to write on board
\item Mention \verb|rnorm|
\item Get the ERROR onto the equation if you have not already
\item mention $n$
\item What is an effect size? 
\end{itemize}

\todo[inline]{Students should pair up and work on doing this with the following rules ...} 
\begin{itemize}
\item You must BOTH end up with the code you come up with.
\item You alert me when you're done or have a question ... [If they are done, they should check their work using lm, then try to simulate a LOGISTIC regression.]
\end{itemize}

\todo[inline]{Note to self: Give the class a 10 minute break by 3pm!} 

\section{Simulate from a linear model: Part 2}

Come together and review how they did. Live code with them the course example using lm to check work. 

\todo[inline]{Discuss: How might this be valuable?} 
And be sure to discuss why this is critical in Bayesian approaches (NO assumptions; you must CHECK and UNDERSTAND your model.}

\section{Simulate from a linear model: Add interactions}

Review this if time allows .... 
\begin{itemize}
\item Intercept only model
\item Adding an interaction to a model ... 
\end{itemize}

\section{Review of course}
\begin{itemize}
\item 3 weeks, 6 classes, we'll get to hierarchical modeling
\item Grading is participation and homework
\item There are TWO homework (end of each of the first 2 weeks). Please do them! Even if you are auditing.
\item No project, you must use a provided dataset
\item Course managed on GitHub; you can submit homework on GitHub or Canvas. 
\item GitHub has wiki with resources ... Review (if time allows)
\item We will use rstanarm, which is a version of Stan -- make sure you have it running before the next class. 
\item Remind me to give you a BREAK in the middle of class
\item Questions?
\end{itemize}


\newpage
\section{Class 2} 
\subsection{Cover interactions! If you have not already ...}


\subsection{What is Bayesian: Posterior}
Go over it.
Give my Star Wars example.
\todo[inline]{Discuss in pairs: Another example!} 

Other examples: Complete separation.... 

\subsection{What is Bayesian: Prior}

Types of priors (informative, non-informative, weakly informative)

Why we will not do prior checks ... \\ Because they are annoying in rstanarm, brms etc. They are easier in raw Stan code. 

EXAMPLE: Show code where likelihood overwhelms prior. 

\subsection{Simulate another example ... and fit it in rstanarm}


\subsection{Review the homework assignment!}
\begin{itemize}
\item How to submit
\item What to do if you get stuck ..
\item A note on using ChatGPT
\end{itemize}

\end{document}


\begin{enumerate}
\item
\end{enumerate}


\begin{itemize}
\item
\end{itemize}
